%%%%%%%%%%%%%%%%%%%%%%%%%%%%%%%%%%%%%%%%%
% Beamer Presentation
% LaTeX Template
% Version 1.0 (10/11/12)
%
% This template has been downloaded from:
% http://www.LaTeXTemplates.com
%
% License:
% CC BY-NC-SA 3.0 (http://creativecommons.org/licenses/by-nc-sa/3.0/)
%
% Modified by Jeremie Gillet in November 2015 to make an OIST Skill Pill template
%
%%%%%%%%%%%%%%%%%%%%%%%%%%%%%%%%%%%%%%%%%

%----------------------------------------------------------------------------------------
%	PACKAGES AND THEMES
%----------------------------------------------------------------------------------------

\documentclass{beamer}

\mode<presentation> {

\usetheme{Madrid}

\definecolor{OISTcolor}{rgb}{0.65,0.16,0.16}
\usecolortheme[named=OISTcolor]{structure}

%\setbeamertemplate{footline} % To remove the footer line in all slides uncomment this line
%\setbeamertemplate{footline}[page number] % To replace the footer line in all slides with a simple slide count uncomment this line

\setbeamertemplate{navigation symbols}{} % To remove the navigation symbols from the bottom of all slides uncomment this line
}

\usepackage{graphicx} % Allows including images
\usepackage{booktabs} % Allows the use of \toprule, \midrule and \bottomrule in tables
\usepackage{textpos} % Use for positioning the Skill Pill logo
\usepackage{fancyvrb}
\usepackage{tikz}
\usepackage{hyperref}
\usepackage{listings}
\usepackage{subfig}
%\usepackage{enumitem}
%\usepackage{calc}

\definecolor{dkgreen}{rgb}{0,0.6,0}
\definecolor{gray}{rgb}{0.5,0.5,0.5}
\definecolor{mauve}{rgb}{0.58,0,0.82}

\newcommand{\red}[1]{{\color{red} #1}}




%~ \lstset{frame=tb,
  %~ language=python,
  %~ aboveskip=3mm,
  %~ belowskip=3mm,
  %~ showstringspaces=false,
  %~ columns=flexible,
  %~ basicstyle={\small\ttfamily},
  %~ numbers=none,
  %~ numberstyle=\tiny\color{gray},
  %~ keywordstyle=\color{blue},
  %~ commentstyle=\color{dkgreen},
  %~ stringstyle=\color{mauve},
  %~ breaklines=true,
  %~ breakatwhitespace=true,
  %~ tabsize=3
%~ }

\usepackage[charsperline=65, theme=default-plain]{jlcode}
\usepackage{tcolorbox}


\lstset{%frame=tb,
  language=julia,
  aboveskip=-1.0mm,
  belowskip=-1.0mm,
  showstringspaces=false,
  columns=flexible,
  %basicstyle={\small\ttfamily},
  numbers=none,
  numberstyle=\tiny\color{gray},
  keywordstyle=\color{blue},
  commentstyle=\color{dkgreen},
  stringstyle=\color{mauve},
  breaklines=true,
  breakatwhitespace=true,
  tabsize=3
}


\newcommand*{\lstitem}[1]{
  \setbox0\hbox{\lstinline{#1}}  
  \item[\usebox0]  
  % \item[\hbox{\lstinline{#1}}]
  \hfill \\
}


\newenvironment{Boxx}{\begin{tcolorbox}[standard jigsaw, opacityframe=0.8, opacityback=0.0,left=2pt,right=2pt,top=0pt,bottom=0pt]}{\end{tcolorbox}}

\lstset{escapeinside={<@}{@>}}


%----------------------------------------------------------------------------------------
%	TITLE PAGE
%----------------------------------------------------------------------------------------

\title[Mini Course]{
Mini Course: Julia\\
\small
Lecture 2: Data Processing and Plotting} % The short title appears at the bottom of every slide, the full title is only on the title page

\author[Juan Polo]{\vspace{-0.75cm} \underline{Juan Polo}, Friederike, Gaston {\tiny $\Bigg\{$ \parbox{2cm}{\centering material taken/based on Ankur Dhar \& James Schloss}$\Bigg\}$}} % Your name
\institute[OIST] % Your institution as it will appear on the bottom of every slide, may be shorthand to save space
{
	\textit{juan.polo@oist.jp} \\ % Your institution for the title page
}
\date{\vspace*{-0.5cm}\\ March 3rd, 2021} % Date, can be changed to a custom date

\begin{document}

\setbeamertemplate{background}{\includegraphics[width=\paperwidth]{images_data/SPbackground.png}} % Adding the background logo

\begin{frame}
\vspace*{1.4cm}
\titlepage % Print the title page as the first slide
\end{frame}

\setbeamertemplate{background}{} % No background logo after title frame

\addtobeamertemplate{frametitle}{}{% Adding the Skill Pill logo on the title screen after title frame
\begin{textblock*}{100mm}(.92\textwidth,-0.9cm)
\includegraphics[height=0.85cm]{images_data/julia.pdf}
\end{textblock*}}

\begin{frame}
  \tableofcontents
\end{frame}
\section{Arrays}



\begin{frame}[fragile]{Installing the plotting packages}

\begin{block}{Plots and Pyplot}

{\color{red} \verb|] add Plots|}

{\color{red} \verb|] add PyPlot|}


{\color{red} \verb|using Plots|}

{\color{red} \verb|import PyPlot|}



\end{block}


\end{frame}






\begin{frame}[fragile]{Creating Arrays}
	Arrays in Julia are defined using square brackets {\color{red}\verb|[ ]|}
	
	By default arrays are row vectors, but can be transposed to column vectors
\pause

	Sequences of numbers can be generated using the {\color{red}\verb|:|}
	 
	Ranges can be defined as {\color{red} \verb|start:step:end|}
	
	To create a vector from a range use the function {\color{red}\verb|collect()|}

\pause
  \begin{Boxx}
  \begin{jllisting}
  x = [1 2 3 4] 			# 1x4 Array{Int64,2}
  y = collect(1:4) 		# 4-element Array{Int64,1}
  z = [1, 2, 3, 4] 		# 4-element Array{Int64,1}
  m = [1 2; 3 4]			# 2x2 Array{Int64,2}
  w=x'						# 4x1 {Int64,Array{Int64,2}}
  array = Int64[]		# 0-element Array{Int64,1}
  array2 = Array{Int64}(undef, 2, 2, 2)	# 2×2×2 Array{Int64,3}
  array3 = [[1,2],[3,4,"5"]] # 2-element Array{Array{Int64,1},1}
	\end{jllisting}
  \end{Boxx}
	
	%~ \begin{lstlisting}
	%~ julia> <@\textcolor{red}{x = [1 2 3 4]}@>  			# 1x4 Array{Int64,2}
	%~ julia> <@\textcolor{red}{y = collect(1:4)}@>  		# 4-element Array{Int64,1}
	%~ julia> <@\textcolor{red}{z = [1, 2, 3, 4] }@> 		  # 4-element Array{Int64,1}
	%~ julia> <@\textcolor{red}{m = [1 2; 3 4] }@> 		    # 2x2 Array{Int64,2}
	%~ julia> <@\textcolor{red}{w=x'}@>					   	  # 4x1 {Int64,Array{Int64,2}}
	%~ julia> <@\textcolor{red}{array = Int64[]}@> 			 # 0-element Array{Int64,1}
	%~ \end{lstlisting}

	\pause
	%~ \vspace*{-0.15cm}
	\fbox{\textbf{\color{red} Please try to create a vector from 0 to 0.9 in steps of 0.1}}

	\pause
	\vspace*{0.15cm}

	\fbox{\parbox{\linewidth}{\textbf{\color{red} ?range} This allows you to create a range without thinking too much. You can combine stop, step and length {\color{red}collect(range(0,stop=10,length=10))}}}
	
\end{frame}

\begin{frame}[fragile]{Array Generation Functions}
	\begin{description}[leftmargin=*]
		\item[zeros(S)] Makes an array of size S filled with zeros  
		\item[ones(S)]  Makes an array of size S filled with ones 
		\item[repeat(A,c,r)] Repeats array A column-wise c times and row-wise r times
		\item[rand(S)] Generates array of size S with random numbers between 0 and 1
		\item[Type\textbf{[]}] Creates empty array of type Type
	\end{description}
	
\end{frame}

\begin{frame}[fragile]{Array Operations}
	Arrays are indexed by square brackets after the array {\color{red}\verb|A[1,2]|}. 
	
\pause	
  \begin{Boxx}
  \begin{jllisting}
		data = rand(50,50)
		data[1,:]
		data[:,1]
		inner = x*y
		outer = y*x
		square = x .* x
	\end{jllisting}
  \end{Boxx}
	%~ \begin{lstlisting}
	%~ julia> <@\textcolor{red}{data = rand(50,50)}@>
	%~ julia> <@\textcolor{red}{data[1,:]}@>
	%~ julia> <@\textcolor{red}{data[:,1]}@>
	%~ julia> <@\textcolor{red}{inner = x*y}@>
	%~ julia> <@\textcolor{red}{outer = y*x}@>
	%~ julia> <@\textcolor{red}{square = x .* x}@>
	%~ \end{lstlisting}

	
	To execute scalar operations on an array, \textbf{you can use a} \verb|.| \textbf{to the operator}. This is also applicable to functions as well.
		
	\pause
	\vspace*{0.5cm}
	\fbox{\textbf{\color{red} Multiply all elements of "x" by a constant "2" (careful with spaces)}}
	\pause
	
	\vspace*{0.5cm}
	
	\centering {\color{red}\Huge \verb|2 .* x|}
	
\end{frame}

\begin{frame}[fragile]{Broadcasting}
	Element-by-element binary operations on arrays of different sizes are possible. Dotted operators, {\color{red}\verb|.+|} or {\color{red}\verb|.*|}, are equivalent to broadcast calls
	
	\vspace{0.25cm}
	
  \begin{Boxx}
  \begin{jllisting}
	data = collect(0:.1:1)
	round.(data)
	f(x) = x+2
	f.(data)
	\end{jllisting}
  \end{Boxx}
	
	
	%~ \begin{lstlisting}
	%~ julia> <@\textcolor{red}{data = collect(0:.1:1)}@>
	%~ julia> <@\textcolor{red}{round.(data)}@>
	%~ julia> <@\textcolor{red}{f(x) = x+2}@>
	%~ julia> <@\textcolor{red}{f.(data)}@>
	%~ \end{lstlisting}
	
		
	\pause
	%~ \vspace*{-0.15cm}
	\fbox{\parbox{\linewidth}{\textbf{\color{red} Create a one dimensional vector ``v'' of length 2 (e.g. elements (1, 2)) and add it to the matrix ``m'' of dim = 2x2}}}
	\pause
  \begin{Boxx}
  \begin{jllisting}
	v = [1, 2]
	m = [1 2; 3 4]
	v .+ m
	\end{jllisting}
  \end{Boxx}
	%~ \begin{lstlisting}
	%~ julia> <@\textcolor{red}{v = [1, 2]}@>
	%~ julia> <@\textcolor{red}{m = [1 2; 3 4]}@>
	%~ julia> <@\textcolor{red}{v .+ m}@>
	%~ \end{lstlisting}
	\pause
	\fbox{\parbox{0.4\linewidth}{\textbf{\color{red} Extra cool things}}}
	
	{\color{red}\Large \verb|data[data .< 0.5]|}
	
\end{frame}

\section{Data Handling}

\begin{frame}[plain]

\centering \Huge \textbf{Data Handling}

\end{frame}

\begin{frame}[fragile]{File Structure}
	To \textbf{p}rint you current \textbf{w}orking \textbf{d}irectory {\color{red} pwd()}

  \begin{Boxx}
  \begin{jllisting}
	pwd()
	# cd("C:\\Users\\M\\Documents\\JuliaStuff")
	readdir()
	\end{jllisting}
  \end{Boxx}
	Change you current directory {\color{red}\verb|cd("String")|} 
	
	List the files within using {\color{red}\verb|readdir()|}
	
\end{frame}

%~ \begin{frame}[fragile]{Writing Scripts}
	%~ As we start to get into more complex commands and chain them together, we can use scripts to automatically execute a series of commands at once. To include these commands into the REPL (read-eval-print loop), we simply use the \verb|include()| command 
	%~ \begin{lstlisting}
	%~ julia> include("Lesson 2.jl")
	%~ julia> randflips[1];
	%~ \end{lstlisting}
	%~ To help debug in these scripts, you can suppress the direct output from any commands with a \verb|;|, and forcibly print values using \verb|prfint()| or \verb|println|.
%~ \end{frame}

\begin{frame}[fragile]{Delimited text}
	
	The simplest data files are delimited text files or CSV files
	
	Requires to load a library {\color{red}DelimitedFiles}
	\vspace*{0.25cm}
	
	\begin{lstlisting}
	julia> using DelimitedFiles
	julia> data = rand(50,50)
	julia> writedlm("Random.txt",data)
	julia> randData = readdlm("Random.txt")
	\end{lstlisting}
	
\vspace*{0.5cm}
	Similarly, any 1D or 2D can be written to a file, with the actual delimiter being based on the file extension used (txt for space and csv for comma).
	
\pause
\vspace*{0.5cm}
	Hierarchical Data Format (HDF5) can be an interesting format to save arrays and data.
	
\end{frame}

\begin{frame}[fragile]{CSV files}
	CSV files are a bit more complex than standard delimited files, including headers or labels. 
	
	A specific library can be loaded {\color{red}CSV}. 
	
	The first row will be taken as the header row (this can be explicitly defined)

  \begin{Boxx}
  \begin{jllisting}
  	using CSV, DataFrames
	data = DataFrame(CSV.File("simplemaps-worldcities-basic.csv"))
	names(data)
	populations = data[:,:pop]
	populations = data.pop
	data[5677,:]
	\end{jllisting}
  \end{Boxx}
	
	%~ \begin{lstlisting}
	%~ julia> <@\textcolor{red}{using CSV, DataFrames}@>
	%~ julia> <@\textcolor{red}{data = DataFrame(CSV.File("simplemaps-worldcities-basic.csv"))}@>
	%~ julia> <@\textcolor{red}{names(data)}@>
	%~ julia> <@\textcolor{red}{populations = data[:,:pop]}@>
	%~ julia> <@\textcolor{red}{populations = data.pop}@>
	%~ julia> <@\textcolor{red}{data[5677,:]}@>
	%~ \end{lstlisting}
	
	%~ julia> using CSV
	%~ julia> data = CSV.read("simplemaps-worldcities-basic.csv")
	%~ julia> names(data)
	%~ julia> populations = data[:pop]
	
	Small trick, you can write ``\verb|data.|'' and press "TAB" twice to see the available columns
	
	
	\tiny
	{\color{gray}\verb|data[data.city .== "Barcelona",:]|
	
	\verb|findall(x->x=="Barcelona",data.city)|}
	
\end{frame}

\begin{frame}[fragile]{Exercises}
	\begin{block}{Exercise 1}
		Creates a file \verb|squares.txt| consisting of the first 5 square numbers
	\end{block}
	\begin{block}{Exercise 2}
		Write a script which creates a new file called \verb|large_cities.txt|.
		The file should contain one line for each of the cities which have a
		population larger than 10,000,000., formatted as follows:\\
		Buenos Aires, Argentina: population 11862073\\
		Sao Paulo, Brazil: population 14433147.5\\
		...
	\end{block}
	
	\pause
	
	{\color{gray} \tiny
	Fast way to look at data using Dataframes \verb|data[data.pop .> 10^7,:]|
	}
	
\end{frame}

\section{Plotting}

\begin{frame}[plain]

\centering \Huge \textbf{Plots}

\end{frame}


\begin{frame}[fragile]{Plots in a Nutshell}
	\begin{minipage}{0.6\linewidth}
		%~ \column{.6\linewidth}
			Let's add and load the general Plots package
		\begin{Boxx}
		\begin{jllisting}
					# inside pkg>
					add Plots
					# inside REPL
					using Plots
					# plot(Plots.fakedata(50))
			\end{jllisting}
		\end{Boxx}
		This will let us call the general Plot commands, in this case \verb|plot| plots 1D data as a line plot.
		
		More information can be found at \url{https://docs.juliaplots.org/latest/}
	\end{minipage}%
	\begin{minipage}{0.4\linewidth}
		\begin{figure}
			\centering
			\includegraphics[width=\textwidth]{images_data/randPlot}
		\end{figure}
	\end{minipage}
\end{frame}


\begin{frame}[fragile]{Plotting Commands}
	\begin{description}[leftmargin=*,align=right]
		\item[scatter(X,Y)] Scatter plot data with XY coordinates
		\item[bar(x,y)] Bar plot following similar rules to plot
		\item[histogram(x,bins=n)] Plots histogram of 1D data in n bins.
		\item[plot($\theta$,r,proj=:polar)] Polar plot of data following r and $\theta$
		\item[heatmap(x,y,z)] Plots heatmap following XY axes with intensity array z
		\item[fakedata(L,S)] Generates random S numbers of series data of length L
		\item[savefig(filename)] Saves a generated plot as an image file
	\end{description}
	
\end{frame}

\begin{frame}[fragile]{Plotting 3D Data}
%%TODO include heatmap, surf, and contour
	\begin{columns}
		\column{.6\textwidth}
		There are a couple of additional options for plotting 3D data:
		\begin{description}[align=right]
			\item[surface(x,y,z)] Draws surface in 3D space
			\item[contour(x,y,z)] Draws contours on 2D plane
			\item[heatmap(x,y,z)] Draws a heatmap on 2D plane
		\end{description}
		The plotting commands \verb|plot|,\verb|scatter|,\verb|bar|, and \verb|heatmap| also can accept 3D data.
		\column{.4\textwidth}
		\begin{figure}
			\centering
			\includegraphics[width=\textwidth]{images_data/surf}
		\end{figure}
	\end{columns}
\end{frame}


\begin{frame}[fragile]{Plotting Vector Data}
	\begin{columns}
	\column{.5\textwidth}
	For vector data the current option is quiver:
	\begin{lstlisting}
	help?> quiver
	search: quiver quiver!
	
	quiver(x,y,quiver=(u,v))
	quiver!(x,y,quiver=(u,v))
	
	Make a quiver (vector field) plot. The ith vector extends from (x[i],y[i]) to (x[i] + u[i], y[i] + v[i]).
	\end{lstlisting}
	\column{.5\textwidth}
	\begin{figure}
		\centering
		\includegraphics[width=\textwidth]{images_data/quiver}
	\end{figure}
\end{columns}

\end{frame}

\begin{frame}[fragile]{Plotting Backends}
	\begin{columns}
	\column{.5\linewidth}
	These commands are agnostic to the plotting backend, meaning they will work with a number of plotting engines in similar fashion. Each backend has pros and cons, but most commonly used are {\color{red} GR and PyPlot}.
	
	\begin{figure}[h]
		\centering
		\includegraphics[width=.75\textwidth]{images_data/randPlotPy}
		\caption{\centering  PyPlot\newline \tiny using PyPlot; PyPlot.plot(rand(10))}
	\end{figure}

	\column{.5\linewidth}
	\begin{figure}[h]
		\centering
		\includegraphics[width=.75\textwidth]{images_data/randPlotPGF}
		\caption{PGFPlots}
	\end{figure}
	\vspace{-11mm}
	\SaveVerb{term}|UnicodePlots.histogram(randn(1000),nbins=15,closed=:left)|
	\begin{figure}[h]
		\centering
		\includegraphics[width=.475\textwidth]{images_data/randPlotUni0}
		\includegraphics[width=.475\textwidth]{images_data/randPlotUni2}
		\caption{
		\centering UnicodePlots\newline 
		\tiny using UnicodePlots; UnicodePlots.lineplot(rand(10))\newline
		%~ \verb|using UnicodePlots; UnicodePlots.histogram(randn(1000) .* 0.1, nbins = 15, closed = :left)|
		\fontsize{5}{5}\selectfont
		\protect\UseVerb{term}
		}
	\end{figure}
	\end{columns}
\end{frame}


\begin{frame}[fragile]{Examples of plots}

		\footnotesize
		You can load one or multiple plot libraries but they might have conflicts. Using ``Plots.plot'' or ``PyPlot.plot'' will force a particular library.
		
		The default backend can be changed by ``backend(:gr)'' or ``backend(:pyplot)''
		
		\pause
		
		\begin{Boxx} 
		\begin{jllisting}
	using Plots; import PyPlot; plt = PyPlot;
	# Import avoids conflicts, however PyPlot.XXX or plt.XXX is need to use that library

	plot(rand(20))		# generates a plot
	plot(rand(20))		# generates a new plot
	plot!(rand(20))	# add line to previous plot
	
	p1 = plot(rand(20)); # the semicolon will prevent plotting
	p2 = plot(rand(20));
	p3 = plot(rand(20));
	plot(p1,p2,p3)	# Automatically generates a layout
	
	# Manually generating a layout
	plot(p1,p2,p3, layout = @layout grid(3,1))	
	plot(p1,p2,p3, layout = @layout [a [b; c]])				
			\end{jllisting}
		\end{Boxx}	


        \begin{picture}(0,0)
            \put(285,120){%
                \includegraphics[width=0.15\textwidth]{images_data/plot_rand}%
            }
        \end{picture}
        
        \begin{picture}(0,0)
            \put(285,70){%
                \includegraphics[width=0.15\textwidth]{images_data/plot_grid_3_1}%
            }
        \end{picture}
        
        \begin{picture}(0,0)
            \put(285,40){%
                \includegraphics[width=0.15\textwidth]{images_data/plot_grid_1_2_1}%
            }
        \end{picture}

\end{frame}


\begin{frame}[fragile]{Examples of plots}

				
		\begin{Boxx} 
		\begin{jllisting}
	using Plots; import PyPlot; plt = PyPlot;
	# Import avoids conflicts, however PyPlot.XXX or plt.XXX is need to use that library

	# Plot with two insets
	plot(1:20)		
	plot!((-10:10).^2, inset = (1, bbox(0.1,0.0,0.4,0.4)), subplot = 2)
	plot!(-(-10:10).^2, inset = (1, bbox(0.6,0.5,0.4,0.4)), subplot = 3)
	
			
			\end{jllisting}
		\end{Boxx}	

\centering \includegraphics[width=0.45\linewidth]{images_data/plot_insets}

\end{frame}


\begin{frame}[fragile]{Examples of plots with PyPlot}



		\begin{Boxx} 
		\begin{jllisting}
	using Plots; import PyPlot; plt = PyPlot;
	# we have to use plt.XXX for the main commands
	
	fig, axs = plt.subplots(ncols=2,nrows=2);
	

	axs[1,1].plot(rand(20));
	axs[1,2].plot(rand(20));
	axs[2,1].plot(rand(20));
	axs[2,2].plot(rand(20));

	axs[1,1].set_title("(a)")	
	axs[1,1].set_xlabel("x_label")
	axs[1,1].set_ylabel("y_label")
	
	
	plt.clf(); # this will clear the figure
		
			\end{jllisting}
		\end{Boxx}	

        \begin{picture}(0,0)
            \put(170,40){%
                \includegraphics[width=0.55\textwidth]{images_data/plot_pyplot}%
            }
        \end{picture}

\end{frame}



\begin{frame}[fragile]{Plotting 3D examples}
%%TODO include heatmap, surf, and contour
	\begin{columns}
		\column{.7\textwidth}
		
		\begin{Boxx}
		\begin{jllisting}
				using Plots
				x = y = collect(-1:0.01:1)
				#surface(x,y,(x,y)->x^2+y^2)
				#heatmap(x,y,(x,y)->x^2+y^2)
				f(x,y) = x^2 + y^2
				surface(x,y,f.(x,y'))
				contour(x,y,f.(x,y'))
				heatmap(x,y,f.(x,y'))
			\end{jllisting}
		\end{Boxx}		
		
		
		\column{.3\textwidth}
		\begin{figure}
			\centering
			\includegraphics[width=\textwidth]{images_data/surf}
		\end{figure}
	\end{columns}
\end{frame}



\begin{frame}[fragile]{Animations}
%%TODO include heatmap, surf, and contour

		\begin{Boxx}
		\begin{jllisting}

	using Plots;
	
	ENV["GKSwstype"]="nul"	#Plots.backends()
	
	anim = Animation()
	
	x = range(0,stop=20*pi,length=200)
	
	p(w) = plot(x,[cos.(x) + cos.(w .* x)], xlims = (0, 20π), ylims = (-2, 2))
	
	for i in range(1,stop=1.25,length=100)
		p(i)
		frame(anim)
	end
	
	gif(anim, "myanimation.gif", fps=15)

			\end{jllisting}
		\end{Boxx}		

\end{frame}



\begin{frame}[fragile]{Examples}
Each of these commands can be expressed in-line with the plotting command or beforehand within a call to the plotting backend.
\begin{description}[leftmargin=*,align=right]
	\item[font(fontname,size)] Defines a Font object with a given size
	\item[size=(X,Y)] Sets size of plot to X by Y pixels
	\item[xlabel=string] Sets X-Y labels to string, also ylabel
	\item[title=string] Sets title to string, also colorbar\_title for heatmap.
	\item[xtickfont=font] Sets the font of x tick marks, also ytickfont,titlefont,guidefont
	\item[left\_margin=length] Sets margin for left side of plot, also top\_margin, bottom\_margin, and right\_margin
	\item[xscale=:log10] Sets x scale to log10, also yscale
\end{description}

\end{frame}


\begin{frame}[fragile]{Exercises}
	\begin{block}{Exercise 3}
		Read data.txt given in the Public Folder and plot the results.
		What do you see?
	\end{block}
	\begin{block}{Exercise 4}
		Plot a histogram of the longitudes and latitudes of the world's cities. 
		{\color{red}\footnotesize 
		
		\verb|using Plots, DataFrames, CSV|
		\verb|data = DataFrame(CSV.File("simplemaps-worldcities-basic.csv"))|
		}
\end{block}

\end{frame}

\begin{frame}[fragile]{Exercise 4 solution}


	
\begin{Boxx}
\begin{jllisting}	

	using Plots, Images, ImageMagick, DataFrames, CSV
	
	data = DataFrame(CSV.File("simplemaps-worldcities-basic.csv"))
	
	histogram(data.lng,bins=200)
	histogram(data.lat)
	
	\end{jllisting}
	
	
	
\end{Boxx}		


\centering \includegraphics[width=0.475\textwidth]{images_data/histogram}%
\includegraphics[width=0.475\linewidth]{images_data/plot_final}%


		
\end{frame}

\begin{frame}[fragile]{My image}


	
\begin{Boxx}
\begin{jllisting}	
 using Plots, Images, ImageMagick, DataFrames, CSV,Shapefile
 
 data = DataFrame(CSV.File("simplemaps-worldcities-basic.csv"))
 
 shp = Shapefile.shapes(Shapefile.Table("ne_110m_coastline.shp"));
 
 p3 = histogram(data.lng,bins=200,xticks = (-200:25:200),label=false);
 p2 = plot(shp,xlim=(-200,200),ylim=(-100,100));
 p2_size = p2.attr.explicit[:size];
 
 p1 = histogram(data.lat,bins=200,ylim = (-100,100),
 	orientation=:horizontal,
 	aspect_ratio=p2_size[1]/p2_size[2],label=false);
 
 plot(p1,p2,p3,p3,
 	layout=@layout [[a{0.2w} b{0.8h}]; [c d]])

	\end{jllisting}


	
	%~ data_full = DataFrame(CSV.File("simplemaps-worldcities-basic.csv"))
	
	%~ img = load("world3.jpg");
	%~ x = range(-200,200,length=size(img,1));
	%~ y = range(0,200,length=size(img,2));
	
	%~ plot(x,y,img)
	
	%~ histogram!(data_full.lng,bins=200,xticks = (-200:25:200),yflip=false,weights= 0.5 .*ones(size(data_full,1)))
	%~ histogram!(data_full.lat.+100,bins=100,xlim = (-200,200),weights= +0.15 .*ones(size(data_full,1)),orientation=:horizontal)


\end{Boxx}		
		

\end{frame}


%~ \begin{frame}{Other libraries}
  %~ \begin{description}
    %~ \item[Seaborn.jl (Python)] A Julia wrapper around the Seaborn data visualization library:
  %~ \end{description}
%~ \end{frame}


\begin{frame}{What is next?}
  \begin{description}
    \item[Last Session (Gaston)] IDEs, Data Structures, Coding Practices, and Algorithms
  \end{description}
\end{frame}
%\begin{frame}[fragile]{Using Fortran and C in Julia}
%	
%	Julia allows you to use other languages (such as Fortran or C) by using the \texttt{ccall} function:
%	
%	\begin{lstlisting}
%	julia> t = ccall((:clock, "libc"), Int32, ())
%	2292761
%	\end{lstlisting}
%	
%	Here, we are calling the \texttt{clock} function from the \texttt{libc} library in C.
%	
%\end{frame}
\end{document}






