%%%%%%%%%%%%%%%%%%%%%%%%%%%%%%%%%%%%%%%%%
% Beamer Presentation
% LaTeX Template
% Version 1.0 (10/11/12)
%
% This template has been downloaded from:
% http://www.LaTeXTemplates.com
%
% License:
% CC BY-NC-SA 3.0 (http://creativecommons.org/licenses/by-nc-sa/3.0/)
%
% Modified by Jeremie Gillet in November 2015 to make an OIST Skill Pill template
%
%%%%%%%%%%%%%%%%%%%%%%%%%%%%%%%%%%%%%%%%%

%----------------------------------------------------------------------------------------
%	PACKAGES AND THEMES
%----------------------------------------------------------------------------------------

\documentclass{beamer}

\mode<presentation> {

\usetheme{Madrid}

\definecolor{OISTcolor}{rgb}{0.65,0.16,0.16}
\usecolortheme[named=OISTcolor]{structure}

%\setbeamertemplate{footline} % To remove the footer line in all slides uncomment this line
%\setbeamertemplate{footline}[page number] % To replace the footer line in all slides with a simple slide count uncomment this line

\setbeamertemplate{navigation symbols}{} % To remove the navigation symbols from the bottom of all slides uncomment this line
}

\usepackage{graphicx} % Allows including images
\usepackage{booktabs} % Allows the use of \toprule, \midrule and \bottomrule in tables
\usepackage{textpos} % Use for positioning the Skill Pill logo
\usepackage{fancyvrb}
\usepackage{tikz}
\usepackage{hyperref}
\usepackage{listings}

\definecolor{dkgreen}{rgb}{0,0.6,0}
\definecolor{gray}{rgb}{0.5,0.5,0.5}
\definecolor{mauve}{rgb}{0.58,0,0.82}

\lstset{frame=tb,
  language=python,
  aboveskip=3mm,
  belowskip=3mm,
  showstringspaces=false,
  columns=flexible,
  basicstyle={\small\ttfamily},
  numbers=none,
  numberstyle=\tiny\color{gray},
  keywordstyle=\color{blue},
  commentstyle=\color{dkgreen},
  stringstyle=\color{mauve},
  breaklines=true,
  breakatwhitespace=true,
  tabsize=3
}

%----------------------------------------------------------------------------------------
%	TITLE PAGE
%----------------------------------------------------------------------------------------

\title[Mini course: Julia]{Mini course: Julia} % The short title appears at the bottom of every slide, the full title is only on the title page
\subtitle{Lecture 1: Introduction}

\author{Friederike, Gaston \& Juan} % Your name
\institute[OIST] % Your institution as it will appear on the bottom of every slide, may be shorthand to save space
{
\textit{friederike.metz@oist.jp} \\ % Your institution for the title page
%\textit{james.schloss@oist.jp} % Your email address
}
\date{February 22, 2021} % Date, can be changed to a custom date

\begin{document}

\setbeamertemplate{background}{\includegraphics[width=\paperwidth]{SPbackground.png}} % Adding the background logo

\begin{frame}
\vspace*{1.4cm}
\titlepage % Print the title page as the first slide
\end{frame}

\setbeamertemplate{background}{} % No background logo after title frame

\addtobeamertemplate{frametitle}{}{% Adding the Skill Pill logo on the title screen after title frame
\begin{textblock*}{100mm}(.92\textwidth,-0.9cm)
\includegraphics[height=0.85cm]{julia.pdf}
\end{textblock*}}

\begin{frame}
  \tableofcontents
\end{frame}
\section{Introduction}



\subsection{Why does Julia exist}
\begin{frame}{Why does Julia exist?}
%  \pause
  \begin{block}{Key Points}
    \begin{enumerate}
    	   	
      \item Low-level languages are fast, but high level languages are readable.
      \item Recent advances in compiler design could bridge these two aspects.
      \item Julia is being openly developed by researchers, for researchers.
    \end{enumerate}
  \end{block}
%\pause
  The old mission statement is available \href{https://julialang.org/blog/2012/02/why-we-created-julia}{\color{blue} here} and some good discussion is available at \href{https://discourse.julialang.org/t/julia-motivation-why-werent-numpy-scipy-numba-good-enough/2236/}{\color{blue} here}. 
%\pause
  \begin{block}{My personal reason}
    A fast, open-source, high level language that is powerful enough to do serious numerical work, while being designed to encourage good code.
  \end{block}

\end{frame}
\subsection{What is wrong with the status quo}
\begin{frame}{The other contenders}
  The typical languages used in science are
  \begin{enumerate}
    \item Python
    \item Matlab
    \item R
  \end{enumerate}

  Once a problem is becoming to big we usually move to
  \begin{enumerate}
    \item C/C++
    \item Fortran
  \end{enumerate}
  This is called the 2+ language problem and Julia is trying to solve that.
\end{frame}

\begin{frame}{Python and Numpy}
  \begin{itemize}
    \item The compilers that exist (Numba) only work on primitive types and not user defined ones.
    \item GIL (Global Interpreter Lock) mask multi-threading hard.
    \item For fast code you need to write it in C.
    \item Numpy is great, but awful syntax for math.
  \end{itemize}
\end{frame}

\begin{frame}{Matlab}
  \begin{itemize}
    \item It costs a lot of money and is not open-source.
    \item Matlab will only be fast for a subset of operations.
    \item Matlab tends to hide the computer from the programmer.
  \end{itemize}
\end{frame}

\begin{frame}{A (biased) performance comparison}
  \center
  \includegraphics[width=\linewidth]{benchmarks}
\end{frame}

\section{Getting started}
\begin{frame}[fragile]{The REPL}
  \begin{block}{The Read-Eval-Print-Loop}
    The REPL is a command-line interface to Julia and is ideal for short experiments.
    \begin{Verbatim}[fontsize=\footnotesize]
               _
   _       _ _(_)_     |  Documentation: https://docs.julialang.org
  (_)     | (_) (_)    |
   _ _   _| |_  __ _   |  Type "?" for help, "]?" for Pkg help.
  | | | | | | |/ _` |  |
  | | |_| | | | (_| |  |  Version 1.5.3 (2020-11-09)
 _/ |\__'_|_|_|\__'_|  |  Official https://julialang.org/ release
|__/                   |

julia> 
\end{Verbatim}

In the \verb|REPL| you can use \verb|?| to switch your \verb|REPL| mode into help mode and get information about functions.
  \end{block}
\end{frame}
\begin{frame}{IDEs}
  There are two main IDEs that are \emph{feature} complete and can be used for Julia. The main one is based on Atom and is called Juno \url{http://junolab.org/}.

\vspace{0.5cm}
  The second one is based on on Visual Studio Code and available at \url{https://marketplace.visualstudio.com/items?itemName=julialang.language-julia}.

\vspace{0.5cm}
  We will not be using these during the course, but you are welcome to try them out
\end{frame}
\begin{frame}[fragile]{Jupyter}
  Jupyter is an interactive web-based client for Python, Julia, R and many other languages.
  It offers a programming environment that is well suited for explorative data analysis or prototyping.
  \begin{block}{Installation}
  \begin{Verbatim}
    julia> ENV["JUPYTER"] = ""
    julia> ] add IJulia
    \end{Verbatim}
  \end{block}
  \begin{block}{Starting a Jupyter session}
    \begin{Verbatim}
    julia> using IJulia
    julia> notebook()
    \end{Verbatim}
  \end{block}
  \begin{block}{JuliaBox}
    There is an online service provided by JuliaComputing at \url{https://juliabox.com} that gives you a cloud version of Jupyter.
  \end{block}
\end{frame}
\section{Language syntax and semantics}
\begin{frame}[fragile]{Variables and datatypes}
  Julia is a dynamic language and so you can simply create variables in any scope.
  \begin{lstlisting}
  x = 1   # x will be of type Int64
  y = 1.0 # y will be of type Float64
  z = 1.0 - 2.0im # z will be an Complex{Float64}
  1//2 # Rational numbers
  "This is a String"
  """
  This is a multiline
  String
  """
  'C' # Character literal
  1.0f0 # Float32 literal
  \end{lstlisting}
  Use \verb|typeof| to check the type of any variable. Variable names can be unicode and so greek symbols can be used.
  In the \verb|REPL| and most editors you can insert them by entering their \LaTeX name and press\verb|[Tab]|.
\end{frame}

\begin{frame}[fragile]{Variables and datatypes}
	
	\begin{block}{Exercise}
		Create a variable $\lambda$ that stores the value of $\pi$.
		Check the type of $\lambda$.
		Look up the docs for the \verb|convert| function.
		Convert $\lambda$ to float.
	\end{block}
	\vfill
	\begin{block}{Solution}
		\begin{lstlisting}[mathescape]
  # \lambda + <tab> = \pi + <tab>
  $\lambda$ = $\pi$
  typeof($\lambda$)
  ?convert
  convert(Float32, $\lambda$)
		\end{lstlisting}
	\end{block}
\end{frame}


\begin{frame}[fragile]{Data structures}
	\begin{lstlisting}
  # Tuple: (item1, item2, ...)
  okinawa = ("Onna", "Nago", "Naha")
  okinawa[1] 
		
  # NamedTuple: (name1 = item1, name2 = item2, ...)
  okinawa = (central = "Onna", north = "Nago", south = "Naha")
  okinawa.north 
		
  # Dictionaries: Dict(key1 => value1, key2 => value2, ...)
  number = Dict("Jeremie" => "12345", "Juan" => nothing)
  number["Juan"] = "4444"
		
  # Arrays: [item1, item2, ...]
  fibonacci = [1, 1, "two", 3, "five", 8, 13]
  fibonacci[3] = 2
	\end{lstlisting}

Julia is 1-based indexing, not 0-based like Python!!!\\
You can add or remove elements using \verb|push!| and \verb|pop!|.
\end{frame}


\begin{frame}[fragile]{Conditionals}
  Julia has all the typical conditionals \verb|if|, \verb|else|, \verb|elseif| which have to end in an \verb|end|.
  Blocks in Julia are not whitespace sensitive and conditionals do not need to be wrapped in round brackets.
  \begin{lstlisting}
  if rand() < 0.5
    println("Gaston is awesome!")
  else
    println("Juan is awesome!")
  end
  
  # Ternary operator: a ? b : c
  rand() < 0.5 ? println("Gaston  is awesome!") : println("Juan  is awesome!")
\end{lstlisting}

\verb|&&| is the \verb|and| operator, $||$ is the \verb|or| operator.
\end{frame}


\begin{frame}[fragile]{Conditionals}
	\begin{block}{Exercise}
	Print an error message if a variable \verb|x| is greater than zero
	\end{block}
	\vfill
	\begin{block}{Solution}
		\begin{lstlisting}[mathescape]
  if (x > 0) error("x cannot be greater than 0") end
	
  # Ternary operator		
  x > 0 ? error("x cannot be greater than 0") : nothing
			
  # Short-circuit evaluation
  (x > 0) && error("x cannot be greater than 0")
		\end{lstlisting}
	\end{block}
\end{frame}

\begin{frame}[fragile]{Loops}
  Julia has \verb|for| and \verb|while| loops. A \verb|while| loop takes a condition and a \verb|for| loop takes an iterator.
  One can use \verb|break| to break out of a loop and \verb|continue| to skip to the next iteration. %It is noteworthy that a \verb|for| loop can take an arbitrary iterator and even desugar tuples.
  \begin{lstlisting}
  for (i, x) in enumerate(['A', 'B', 'C'])
    if x == 'B'
      continue
    end
    println(i)
  end

  while true
    rand() < 0.1 ? break : println("You are trapped!")
  end
  \end{lstlisting}
\end{frame}


\begin{frame}[fragile]{Loops}
\begin{block}{Exercise}
	Create a dictionary that holds integers (up to the value 100) and their squares as key, value pairs.
\end{block}
\vfill
\begin{block}{Solution}
	\begin{lstlisting}
  squares = Dict()
  for i in 1:100
    squares[i] = i^2
  end
  println(squares)
  
  # Array comprehension
  squares_array = [i^2 for i in 1:100]
	\end{lstlisting}
\end{block}
\end{frame}



\begin{frame}[fragile]{Functions and lambdas}
  Julia uses functions to organize operations. Every function is compiled for the combination of input parameters.
  \begin{lstlisting}
  function f(x, y)
    return x + y
  end

  g(x) = x^2
  h = x -> 1/x
  
  # map takes a function as input and applies that function to every element of the data structure you pass it.
  map(lowercase, ['A', 'B', 'C'])
  map(x->x+2, [1, 2, 3])
  \end{lstlisting}
By convention, functions followed by ! alter their contents and functions lacking ! do not, e.g. \verb|sort()| vs. \verb|sort!()|.
\end{frame}

\begin{frame}[fragile]{Types and multiple dispatch}
	The \verb|::| operator annotates the type of variables. For example, \verb|x::Int8| means that the value of \verb|x| is an instance of \verb|Int8|. There are abstract types (Integer, Number) and concrete types (Int8, Float64) and their relationships form a type graph.
	\vspace*{5mm}
	
	In Julia a function is a set of multiple methods. Each method is defined for particular argument types and comes with its own implementation. When you call a function the most specific method is executed (dispatched).
	\begin{lstlisting}
  h(x::Number) = println("x is most definitly a number.")

  h(x::Integer) = println("x is an integer")
  
  h(x::Int8) = println("Specific method for Int8")
	\end{lstlisting}
Multiple dispatch makes your code generic and fast!
\end{frame}


\begin{frame}[fragile]{Types and structs}
 You can create your own types using \verb|struct|.
  \begin{lstlisting}
  abstract type Coordinate end
  
  mutable struct Cartesian <: Coordinate
    position::Tuple{Float64, Float64}
  end
  
  struct Polar <: Coordinate
    radius::Float64
    phi::Real
  end
  
  point = Polar(1,pi)
  point.radius
  \end{lstlisting}
Keep in mind:~Julia is not an Object-Oriented programming language so functions do not belong to an object!
\end{frame}

\begin{frame}[fragile]{Types}
	\begin{block}{Exercise}
		Create a struct \verb|Point| with two field names \verb|x,y| both being of type Real. Implement a function \verb|add| that takes two arguments of type \verb|Point| and returns the sum of their x's and the difference of their y's.
	\end{block}
	\vfill
	\begin{block}{Solution}
		\begin{lstlisting}
  struct Point
    x::Real
    y::Real
  end
  
  function add(arg1::Point, arg2::Point)
    return arg1.x + arg2.x, arg1.y - arg2.y
  end
		\end{lstlisting}
	\end{block}
\end{frame}



\begin{frame}[fragile]{Running julia scripts}
 \begin{lstlisting}
  # in your command prompt or terminal
  julia script.jl
  
  # inside the REPL
  include("script.jl")
\end{lstlisting}
Working from the REPL is preferred since all methods inside the script are compiled the first time and stay available as long as the session is not closed.\\

\vspace{4mm}
Try to break your programs down into lots of small specialized functions to really take advantage of the julia compiler and its speed-up.
\end{frame}


\begin{frame}[fragile]{Modules}
If your julia files are becomming to large, organize them into modules (namespaces). Module names are capitalised and you can nest modules as well.

  \begin{lstlisting}
  module MyModule
    export f

    g() = "Internal function"
    f() = println(g())
  end
  
  # Loading the module in a different file
  using Main.MyModule
  \end{lstlisting}
Additional advantage: Modules can be precompiled and saved to hard-drive which means precompiled code won't disappear when closing the REPL.
\end{frame}


\section{The ecosystem}
\begin{frame}[fragile]{Installing packages}
  Julia has an inbuilt package manager called \verb|Pkg|. You can access it from the REPL by entering \verb|]|.
  For example, to install and load the Julia package \verb|Distributions.jl| in the \verb|REPL| just run:

  \begin{lstlisting}
  # Installing a package
  ] add Distributions
  # Loading it afterwards
  using Distributions
  \end{lstlisting}

  A few other commands:
  \begin{lstlisting}
  # Updating all installed packages
  ] update
  # What packages are installed?
  ] status
  \end{lstlisting}
  Look \href{https://julialang.org/packages/}{\color{blue} here} for more information about Julia packages.
\end{frame}
\begin{frame}{Resources}
\begin{description}
	\item[Documentation] \url{https://docs.julialang.org/en/v1.5/}
	\item[Forum] \url{https://discourse.julialang.org}
	\item[GitHub] \url{https://github.com/JuliaLang/julia}
	\item[Slack] \url{https://julialang.org/slack/}
	\item[YouTube] \url{https://www.youtube.com/user/JuliaLanguage}
	\item[Learning] \url{https://julialang.org/learning/}
\end{description}
\vfill
OIST also has a Slack workspace: julia@oist
\end{frame}
\begin{frame}{What is next?}
  \begin{block}{Question?!}
    What do you want to hear/learn more about?
  \end{block}
  \begin{description}
    \item[Wednesday] How to use Julia for computation and plotting
    \item[Friday] Problem sets and ???
  \end{description}
\end{frame}

\end{document}
